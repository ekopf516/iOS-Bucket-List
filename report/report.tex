\title{iOS Mini-App: UVa Bucket List}
\author{
        Richard Dizon \& Emma Kopf\\
        CS 4720: Team Mutsu\\
        University of Virginia\\
}
\date{October 7, 2016}

\documentclass[12pt]{article}

\begin{document}
\maketitle

\section{Instructions and Information}
\noindent Stuff about instructions and information here. Will re-do this mess later in a branch. Ignore anything below because it's old.

\section{Lessons Learned}
\textbf{Emma:} Previously, I didn't have experience developing in Android because I took 2110 after the Android Project was scrapped from the syllabus. I had to get used to a lot of new aspects of coding/Java that I never dealt with before, such as XML files, intents, activities, fragments, and etc. Through this project, I better understand the underlying aspects that go into creating something as simple as a button on a screen, and all the different steps it takes to go from having said button pop up on the screen, have a click registered, and transfer data. The trials and errors I went through while working on my app taught me a lot about streamlining development as well as writing generally good, understandable code. It was definitely interesting working on this project starting with nothing. In 1110, making the game wasn't too difficult because all of the style/formatting was given to us. We just had to implement some code that we already wrote in lab or for homework. \break

\noindent I also had to do a lot of work getting to understand GitHub and how to push/pull from remote repositories. Previously, I only ever uploaded completed projects to GitHub and never went back to expand on them. We definitely had a lot of issues with pushing/pulling branches and Richard had to do a lot of fixing/merging.\break

\noindent \textbf{Richard:} On the flip-side, I had to quickly remember everything that I learned back from first-year when I took 2110 when it still had the Android project, but the transition wasn't nearly as smooth as I thought it would be. After having done a lot of web development and project maintenance over the summer, it was weird to go back and start a project from scratch like this. Knowing XML this time around eased things, but for sure it's definitely much easier to add features to something that's already built as opposed to building things from the ground up. Programming in Java after two-years wasn't as bad as I expected either. The whole process wasn't \textit{the most} fun but it's definitely a good learning experience.\break

\noindent I specifically remember struggling a lot back then with working with version control across four team members, but now that I had come in with a lot of Git experience, naturally using it was the best option, even working with just one other partner. I hadn't realized how complicated version control can be before I started explaining to Emma how to manage branching, committing and merging safely, but we both definitely benefited from that experience.

\section{Sources Used}
\begin{enumerate}
\item Lynda Tutorial - iOS 10 App Development Essentials 1: Create Your First App
\end{enumerate}

\end{document}
